\documentclass{article}

% ----------- core mathematics -----------
\usepackage{amsmath}   % align, split, cases, etc.
\usepackage{amssymb}   % \mathbb, \mathcal, \leqslant, \geqslant ...
\usepackage{amsfonts}  % blackboard bold, fraktur if desired
\usepackage{graphicx}  % for future figures, even if none are included yet
\usepackage{listings}  % for code examples
\usepackage{hyperref}  % clickable cross-refs; load last
\usepackage{mdframed}  % for boxed content

% ----------- tables & arrays ------------
\usepackage{array}     % extended column specifiers in tabular
\usepackage{booktabs}  % nicer horizontal rules (optional; you may keep \hline)
\usepackage{multirow}  % multi-row cells if you extend the tables later

% ----------- layout & floats ------------
\usepackage{geometry}  % easy margin control (defaults are fine; optional)
\usepackage{caption}   % better caption spacing for tables/figures

\title{Surprisingness }
\order{4}

\begin{document}

\section{Understanding Surprisingness}

Surprisingness (or "interestingness") in the Hyperon Pattern Miner is implemented in \texttt{utils/MinerUtils.metta} by the I-Surprisingness measure. It compares each pattern's empirical probability against an expected probability under a simple independence-based null model, then normalizes the largest deviation.

\subsection{Universe Size and Empirical Probability}

First, the miner computes the total number of ground atoms:

\begin{verbatim}
;; Count every atom in the database
(= (universe-size)
   (let $u (collapse (match $dbspace $x 1))
       (tuple-count $u)))
\end{verbatim}

Given a pattern's support count, its empirical probability is computed as:

\begin{verbatim}
;; P_obs = support / universe-size
(= (prob $count)
   (// $count (universe-size)))
\end{verbatim}

This defines
\[
  P_{\mathrm{obs}}
  = \frac{\mathrm{support}}{\text{universe-size}}.
\]
\subsection{I-Surprisingness for 2 and 3-Clause Patterns}

The core logic lives in:

\begin{verbatim}
(= (iSurprisingness $pattern)
  (case $pattern
    ;; 2-clause patterns
    ((candidate (, $p1 $p2) $cnt)
     (let* (($pp1   (prob (count $p1)))
            ($pp2   (prob (count $p2)))
            ($exp   (* $pp1 $pp2))
            ($obs   (prob $cnt)))
       (// (max (- $obs $exp)
                (- $exp $obs))
           $obs)))
    ;; 3-clause patterns
    ((candidate (, $p1 $p2 $p3) $cnt)
     (let* (($pp1    (prob (count $p1)))
            ($pp2    (prob (count $p2)))
            ($pp3    (prob (count $p3)))
            ($pp1p2  (prob (count (, $p1 $p2))))
            ($pp1p3  (prob (count (, $p1 $p3))))
            ($pp2p3  (prob (count (, $p2 $p3))))
            ($maxP   (max (* $pp1p2 $pp3)
                          (max (* $pp1p3 $pp2)
                               (max (* $pp2p3 $pp1)
                                    (* $pp1 (* $pp2 $pp3))))))
            ($minP   (min (* $pp1p2 $pp3)
                          (min (* $pp1p3 $pp2)
                               (min (* $pp2p3 $pp1)
                                    (* $pp1 (* $pp2 $pp3))))))
            ($obs    (prob $cnt)))
       (// (max (- $obs $maxP)
                (- $minP $obs))
           $obs)))
    ;; fallback
    ($_ 0)))
\end{verbatim}

For 2-clause patterns:
\[
  I = \frac{\max\{\,P_{\mathrm{obs}}-P_{\mathrm{exp}},\;P_{\mathrm{exp}}-P_{\mathrm{obs}}\}}{P_{\mathrm{obs}}},
  \quad
  P_{\mathrm{exp}} = P(c_{1})\,P(c_{2}).
\]

For 3-clause patterns, consider all splits into two independent blocks, compute each block's product probability, take the maximum and minimum of those four values, and then:
\[
  I = \frac{\max\{\,P_{\mathrm{obs}}-P_{\mathrm{max}},\;P_{\mathrm{min}}-P_{\mathrm{obs}}\}}{P_{\mathrm{obs}}}.
\]

Patterns with other clause counts default to zero.

\subsection{Filtering by Surprisingness Threshold}

To filter patterns exceeding a threshold \texttt{highsurp}, use:

\begin{verbatim}
;; Returns true if I-Surprisingness > highsurp
(= (isurp? ($pattern $cnt))
   (if (> (iSurprisingness (candidate $pattern $cnt))
          (highsurp))
       true
       false))
\end{verbatim}

Or call the high-level function:

\begin{verbatim}
!(import! &db experiments:data:ugly_man_sodaDrinker)
!(import! &self  chaining:dtl:backward:curried)
!(import! &temp   experiments:miner:miner-rules)
!(import! &self  experiments:miner:system-proofs)
!(bind! &kb (new-space))
(=(min-sup) 6)
(=(surp-mode ) isurp)
(=(db-ratio) 0.5)
! (cog-mine &db &kb (min-sup) (surp-mode) (db-ratio))
\end{verbatim}

This returns only those patterns whose I-Surprisingness exceeds \texttt{highsurp}.







\end{document}