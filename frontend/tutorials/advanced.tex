\documentclass{article}
\usepackage{amsmath}
\usepackage{amssymb}
\usepackage{graphicx}
\usepackage{listings}
\usepackage{hyperref}

\title{Advanced Examples and Use Cases}
\order{7}

\begin{document}


\section{Advanced Examples and Use Cases}

The \texttt{hyperon-miner} repository also contains code corresponding to a few richer pattern mining scenarios, illustrating how you can adapt and extend the basic mining pipeline for different domains and custom requirements.

\subsection{Mining Attribute Co-occurrence in the Sample Dataset}

The file \texttt{data/sample.metta} defines a toy population of ''humans'' with attributes such as \texttt{man}/\texttt{woman}, \texttt{ugly}/\texttt{beautiful}, and \texttt{sodadrinker}.  You can mine this dataset for surprising attribute co-occurrences:

\begin{verbatim}
;; 1. Start a fresh AtomSpace and load the sample data
(def atom-space (atomspace.new))
(load-file "data/sample.metta")
;; 2. Mine up to 2-clause patterns, requiring at least 5 matches
(def opts {:graph       atom-space
           :max-clauses 2
           :min-freq    5
           :measure     i-surprisingness})
(def results (miner.mine-patterns opts))
;; 3. Display the top patterns
(take 5 results)
\end{verbatim}

Expected output might include:

\begin{verbatim}
(and 
  (inherit ($X) sodadrinker)
  (inherit ($X) man))    ; freq: 12, I-surprisingness: +1.2
(and 
  (inherit ($X) sodadrinker)
  (inherit ($X) ugly))   ; freq:  8, I-surprisingness: +0.8
\end{verbatim}

\subsection{Specialized Pipeline via frequent-pattern-miner.metta}

The script \texttt{experiments/rules/frequent-pattern-miner.metta} automates data loading, mining frequent patterns:

\begin{verbatim}
! (register-module! ../../../hyperon-miner)
! (import! &dbspace   hyperon-miner:experiments:data:ugly_man_sodaDrinker)
! (import! &self hyperon-miner:experiments:rules:build-specialization)
! (import! &self hyperon-miner:experiments:rules:candidate-patterns)
! (import! &self hyperon-miner:experiments:rules:conjunction-expansion)
! (frequent-pattern-miner &kb &dbspace &specspace &cndpspace &aptrnspace &conjspace 10 1)
\end{verbatim}

Running this yields a list of the most frequent patterns in the dataset.

\subsection{Conjunction Expansion}

To experiment with pattern-joining/conjunction strategies, call the expand_conjunction function in the conjunction module directly:

\begin{verbatim}
! (import! &self hyperon-miner:experiments:rules:candidate-patterns)
! (import! &self hyperon-miner:experiments:rules:conjunction-expansion)
! (import! &conj-exp) #python module 
! (= (minsup) 5)
! (= (depth) Z)
;; 1. Generate 1-clause candidates above support threshold 5
!(candidatePatterns &dbspace &specspace (minsup) &cndpspace)
;; 2. Expand to 2-clause patterns with custom rules
!(expand_conjunction $conjunct $pattern &dbspace (minsup) 2 False) 
;;conjuct and #pattern are candidate patterns
;; alternatively if we want to perform conjunction expansion on all the candidate patterns for however many clauses we want all at once
!(do-conjunct &dbspace $cndpspace $conjunct (minsup) (depth)) 
;;conjuct is our starting pattern and we conjoin it with all the patterns in &cndpspace recursively
\end{verbatim}

This lets you expand two patterns into conjunction by creating a common variable between them and control which variable merges are allowed before support filtering.

\subsection{Type-Constrained Mining via Dependent Types}

The \texttt{dependent-types/} folder shows how to enforce structural constraints on patterns:

\begin{verbatim}
;; 1. Load the dependent-type miner module
(load-file "dependent-types/MinerCurriedDTL.metta")
;; 2. Instantiate a type-aware miner
(def typed-miner (MinerCurriedDTL atom-space))
;; 3. Mine patterns respecting DTL rules
(def dtl-results
  (typed-miner.mine-patterns
    :max-clauses 3
    :min-freq    4
    :measure     i-surprisingness))
\end{verbatim}

Only patterns meeting your dependent-type invariants will survive.

\subsection{JSD-Based Surprisingness}

You can use a Jensen-Shannon Divergence measure from \texttt{experiments/rules/jsd-surpr}:

\begin{verbatim}
!(import! &db experiments:data:ugly_man_sodaDrinker)
!(import! &self  chaining:dtl:backward:curried)
!(import! &temp   experiments:miner:miner-rules)
!(import! &self  experiments:miner:system-proofs)
!(bind! &kb (new-space))
(=(min-sup) 6)
(=(surp-mode ) jsdsurp)
(=(db-ratio) 0.5)
! (cog-mine &db &kb (min-sup) (surp-mode) (db-ratio))
\end{verbatim}

This surfaces clause pairs whose joint distribution diverges most from independence.

\subsection{Empirical Truth-Value (EMPTV) Scoring}

When your AtomSpace uses STV atoms, convert pattern probabilities into STV judgments:

\begin{verbatim}
;; 1. Import the EMPTV rule
(import! &etv-ex hyperon-miner:experiments:rules:emp-tv)
;; 2. Mine 1-clause patterns with support >= 8
(=(min-sup) 8)
! (cog-mine &db &kb (min-sup))
;; 3. Tag each with an STV
\end{verbatim}

You will get a list of triples:

\begin{verbatim}
(pattern)   support-count   (stv pattern probability confidence)
\end{verbatim}

allowing you to filter by both frequency and subjective confidence in one pass.



\end{document}