\documentclass{article}
\usepackage{amsmath}
\usepackage{amssymb}
\usepackage{graphicx}
\usepackage{listings}
\usepackage{hyperref}

\title{Your First Pattern Mining Task}
\order{6}

\begin{document}


\section{Your First Pattern Mining Task}

So let's use this process to actually do something!  In this section we run the Hyperon Pattern Miner end-to-end on the provided \texttt{experiments/data/sample-data.metta} dataset, using both the high-level \texttt{miner} API and the example \texttt{frequent-pattern-miner.metta} script.

\subsection{Prepare the AtomSpace and Load Sample Data}

\begin{verbatim}
; Import and instantiate AtomSpace and Load the toy dataset (15 human and STV atoms)
! (register-module! ../../../hyperon-miner)
! (import! &dbspace hyperon-miner:experiments:data:sample-data)
\end{verbatim}

The file \texttt{experiments/data/sample.metta} defines a small population of individuals with labels such as \texttt{human}, \texttt{ugly}, and \texttt{sodadrinker}, along with STV (subjective truth value) atoms.

\subsection{Run the High-Level Miner}

First, import the core mining functions:

\begin{verbatim}
! (import! &self hyperon-miner:experiments:rules:build-specialization)
! (import! &self hyperon-miner:experiments:rules:candidate-patterns)
! (import! &self hyperon-miner:experiments:rules:conjunction-expansion)
! (import! &self hyperon-miner:experiments:utils:common-utils)
\end{verbatim}

Then invoke the miner for patterns up to 2 clauses with minimum support 10:

\begin{verbatim}
! (frequent-pattern-miner &kb &dbspace &specspace &cndpspace &aptrnspace &conjspace 10 2)
;; &kb your ;;knowledge base (AtomSpace)
;; &dbspace ;;your AtomSpace database
;; &specspace ;; space for storing specialized patterns
;; &cndpspace ;; space for storing candidate patterns
;; &aptrnspace ;; space for storing abstract patterns
;; &conjspace ;; space for storing conjunctions (multi-clause patterns)
;; 10 ;; minimum support
;; 2 ;; maximum number of clauses per pattern
\end{verbatim}

You should see output such as:

\begin{verbatim}
Pattern: (supportOf (inherit $X human) 15)
Pattern: (supportOf (inherit $X sodadrinker) 12)
Pattern: (supportOf (, (inherit $X ugly) (inherit $X sodadrinker)) 8)
...
\end{verbatim}

\subsection{Filter by Surprisingness}

To retain only those patterns whose I-Surprisingness exceeds 0.2, call:

\begin{verbatim}
(def surp-results
  (miner-surprising atom-space
                    10     ;; minsup
                    2      ;; max-clauses
                    0.2))  ;; highsurp threshold
(for-each surp-results
  (lambda (c)
    (print (get-pattern c) (get-cnt c))))
\end{verbatim}

This surfaces only patterns whose co-occurrence deviates significantly from the independence model.

\subsection{Using the Provided Script}

The repository includes an example pipeline in \texttt{experiments/rules/frequent-pattern-miner.metta}. From your shell, run:

\begin{verbatim}
meTTa -f experiments/rules/frequent-pattern-miner.metta
\end{verbatim}

This script will:\\

\begin{enumerate}
  \item Register the \texttt{hyperon-miner} module.
  \item Import the \texttt{data/sample.metta} into its DB space.
  \item Execute \texttt{(frequent-pattern-miner 5 2)} with \texttt{minsup=5} and \texttt{depth=2}.
  \item Print all discovered 1- and 2-clause patterns to the REPL.
\end{enumerate}

With these steps you can quickly mine and inspect both frequent and surprising patterns on any AtomSpace dataset.


\end{document}