\documentclass{article}
\usepackage{amsmath}
\usepackage{amssymb}
\usepackage{graphicx}
\usepackage{listings}
\usepackage{hyperref}

\title{The Pattern Mining Process}
\order{5}

\begin{document}


\section{The Pattern Mining Process}

Now we jump into the practical software aspect.   How does pattern mining actually work, leveraging all the contexts introduced above, in the Hyperon framework today?

The Hyperon Pattern Miner is implemented entirely in MeTTa, with its core logic split between two modules:

\begin{itemize}
  \item \texttt{match/MinerMatch.metta} -- orchestrates the multi-stage mining loop
  \item \texttt{utils/MinerUtils.metta} -- provides low-level support routines (link extraction, counting, surprisingness, etc.)
\end{itemize}

Below is a step-by-step walkthrough of how \texttt{(miner \$db \$ms \$depth)} discovers frequent--and optionally surprising--patterns in your AtomSpace.

\subsection{Entry Point: frequent-pattern-miner and miner-surprising}

In \texttt{match/MinerMatch.metta} you will find two top-level definitions:

\begin{verbatim}
;; Return all frequent patterns up to `depth' clauses
(= (frequency-pattern-miner $kb $dbspace $specspace $cndpspace $aptrnspace $conjspace $minsup $depth)
    (superpose (
            (abstract-pattern $dbspace $minsup $aptrnspace) ; prevent patterns that dont have a support from being speciallized
            (match $aptrnspace (AbstractPattern $pattern) (build-specialization $pattern $dbspace $specspace)) ; specializa abstract patterns
            (candidatePattern $dbspace $specspace $minsup $cndpspace); get candidate patterns
            (match $cndpspace (CandidatePattern $pattern) (let $conj (do-conjunct $dbspace $cndpspace (replacev $pattern) $minsup (fromNumber $depth)) (if (== $conj ()) () (add-atom $conjspace $conj)))) ; store candidate patterns
            (formatter $kb $conjspace $dbspace)
        ))
)
;; Return only those patterns whose I-Surprisingness > highsurp
(= (miner-surprising $db $ms $depth $highsurp)
  (let* (($cptrn (miner $db $ms $depth))
         ($isurp (iSurprisingness $cptrn)))
    (if (> $isurp $highsurp)
        (surp (get-pattern $cptrn) $isurp)
        (superpose ()))))
\end{verbatim}

Here:\\

\begin{itemize}
  \item \texttt{\$dbspace} your AtomSpace database
  \item \texttt{\$kb} your knowledge base (AtomSpace)
  \item \texttt{\$specspace} space for storing specialized patterns
  \item \texttt{\$cndpspace} space for storing candidate patterns
  \item \texttt{\$aptrnspace} space for storing abstract patterns
  \item \texttt{\$conjspace} space for storing conjunctions (multi-clause patterns)
  \item \texttt{\$minsup} minimum support threshold
  \item \texttt{\$depth} maximum number of clauses per pattern
  \item \texttt{highsurp} surprisingness threshold (for \texttt{miner-surprising})
\end{itemize}

\subsection{Stage 1: Extract Abstract Patterns with abstract-pattern}

Abstract patterns are the raw link atoms in your AtomSpace, each turned into a 1-clause template by adding \texttt{VariableNode}s to the link.  In \texttt{experiments/rules/frequent-pattern-miner.metta}:

\begin{verbatim}
(= (abstract-pattern $dbspace $minsup $aptrnspace)
            (let* (
                    ($linkunique (unique (match $dbspace ($link $x $y) $link)))
                    ($result (sup-eval $dbspace ($linkunique $z $t) $minsup))
                )
            (if (== $result True)
                (superpose (
                        (remove-atom $aptrnspace (AbstractPattern ($linkunique Z (S Z))))
                        (add-atom $aptrnspace (AbstractPattern ($linkunique Z (S Z))))
                    ))
            empty
        )
)
)
\end{verbatim}

This produces skeletons like \texttt{(drink $x $y)} or \texttt{(drink Z (S Z))}.

\subsection{Stage 2: Generate Specializations or Potential Candidates via build-specialization}

The function \texttt{build-specialization} handles the specialization of the abstract patterns:

\begin{verbatim}
(= (build-specialization ($link $x $y) $dbspace $specspace) 
            (let*
                (
                    ( ($link $x1 $y1) (replacev ($link $x $y)))                            ; replacev is a function that replaces index with variables
                    ( ($subx $suby) (match $dbspace ($link $x1 $y1) ($x1 $y1)))            ; getting the substitution /valuation of the pattern
                    ($shabx (depth-handler $subx))                                    ;handle depth in the x node
                    ($shaby (depth-handler $suby))                                    ; handle depth in the y node
                    ($spec1 (replace (SpecializationOf ($link $shabx $y1) ($link $x $y)))) ;build the specialization and change the variables to indexs 
                    ($spec2 (replace (SpecializationOf ($link $x1 $shaby) ($link $x $y)))) ;build the specialization and change the variables to indexs
                )
            (
                superpose (
                    (remove-atom $specspace $spec1) ; remove the old specialization to avoid redundancy
                    (add-atom $specspace $spec1) ; add the new specialization
                    (remove-atom $specspace $spec2)
                    (add-atom $specspace $spec2)
                )
        )
)
)
\end{verbatim}

\texttt{specialize} builds specializations by replacing variables with groundings.

\subsection{Stage 3: Counting and Support Filtering/ Candidate Pattern}

The function \texttt{candidatePattern} handles filtertion of specializations using minimum support:

\begin{verbatim}
(= (candidatePattern $dbspace $spezspace $minsup $cndpspace)
    (match $spezspace (SpecializationOf $specialized $pattern)
        (let* (
                ( $specializedvar (replacev $specialized))
                ($result (sup-eval $dbspace $specializedvar $minsup))
            )
        (if (== $result True)
            (add-atom $cndpspace (CandidatePattern $specialized))
            $result
        )
)
)
)
\end{verbatim}

This filters out any specializations that do not meet the minimum support threshold.

\subsection{Stage 4: Conjunction Expansion via do-conjunct}

The function \texttt{do-conjunct} handles the conjunction of candidate patterns:

\begin{verbatim}
(= (do-conjunct $db $cndb $conjunct $ms Z)
    (let* (
        ($matches (match $cndb (CandidatePattern $pattern) (replacev $pattern)))
        ($listconj (expand_conjunction $conjunct $matches $db $ms 2 False))
        ($debconj (replace $listconj))
        ($fitlered (remove_conjuncts_with_redundant_clauses $debconj))
    )
    $fitlered
    )
)
(= (do-conjunct $db $cndb $conjunct $ms (S $K))
    (let* (
        ($matches (match $cndb (CandidatePattern $pattern) (replacev $pattern)))
        ($listconj (expand_conjunction $conjunct $matches $db $ms 2 False))
        ($debconj (replace $listconj))
        ($fitlered (remove_conjuncts_with_redundant_clauses $debconj))
        ($varconj (replacev $fitlered))
        ($conj (do-conjunct $db $cndb $varconj $ms $K))
    )
    $conj
    )
)
\end{verbatim}


This combines candidate patterns into conjunctions recursively for any amount of required clauses if they by creating common variables between them.

\subsection{Stage 5: Scoring by I-Surprisingness}

The surprisingness measure is implemented as:

\begin{verbatim}
(= (iSurprisingness $pattern)
  (case $pattern
    ;; 2-clause
    ((candidate (, $p1 $p2) $cnt)
     (let* (($pp1   (prob (count $p1)))
            ($pp2   (prob (count $p2)))
            ($exp   (* $pp1 $pp2))
            ($obs   (prob $cnt)))
       (// (max (- $obs $exp)
                (- $exp $obs))
           $obs)))
    ;; 3-clause
    ((candidate (, $p1 $p2 $p3) $cnt)
     ;; compute pairwise and triple-block expectations...
     ...)
    ;; other arities
    ($_ 0)))
\end{verbatim}

Here \texttt{prob} divides by \texttt{universe-size}, and the numerator measures the largest deviation from the null model.

\subsection{Stage 6: Putting It All Together}

A full mining call looks like:

\begin{verbatim}
!(import! &db experiments:data:ugly_man_sodaDrinker)
!(import! &self  chaining:dtl:backward:curried)
!(import! &temp   experiments:miner:miner-rules)
!(import! &self  experiments:miner:system-proofs)
!(bind! &kb (new-space))
(=(min-sup) 6)
(=(surp-mode ) jsdsurp)
(=(db-ratio) 0.5)
! (cog-mine &db &kb (min-sup) (surp-mode) (db-ratio))
\end{verbatim}

This single invocation will:

\begin{enumerate}
  \item Initialize the miner and index link nodes
  \item Generate and filter 1-clause candidates
  \item Recursively expand to 2- and 3-clause patterns
  \item Score patterns by I-Surprisingness (if using \texttt{miner-surprising})
  \item Return a stream of \texttt{(pattern, score)} entries
\end{enumerate}

With this mapping to the codebase, you can see exactly how Hyperon Pattern Miner transforms raw AtomSpace data into ranked pattern templates.

\end{document}